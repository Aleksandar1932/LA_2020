\documentclass[11pt]{article}

    \usepackage[breakable]{tcolorbox}
    \usepackage{parskip} % Stop auto-indenting (to mimic markdown behaviour)
    
    \usepackage{iftex}
    \ifPDFTeX
    	\usepackage[T1]{fontenc}
    	\usepackage{mathpazo}
    \else
    	\usepackage{fontspec}
    \fi

    % Basic figure setup, for now with no caption control since it's done
    % automatically by Pandoc (which extracts ![](path) syntax from Markdown).
    \usepackage{graphicx}
    % Maintain compatibility with old templates. Remove in nbconvert 6.0
    \let\Oldincludegraphics\includegraphics
    % Ensure that by default, figures have no caption (until we provide a
    % proper Figure object with a Caption API and a way to capture that
    % in the conversion process - todo).
    \usepackage{caption}
    \DeclareCaptionFormat{nocaption}{}
    \captionsetup{format=nocaption,aboveskip=0pt,belowskip=0pt}

    \usepackage{float}
    \floatplacement{figure}{H} % forces figures to be placed at the correct location
    \usepackage{xcolor} % Allow colors to be defined
    \usepackage{enumerate} % Needed for markdown enumerations to work
    \usepackage{geometry} % Used to adjust the document margins
    \usepackage{amsmath} % Equations
    \usepackage{amssymb} % Equations
    \usepackage{textcomp} % defines textquotesingle
    % Hack from http://tex.stackexchange.com/a/47451/13684:
    \AtBeginDocument{%
        \def\PYZsq{\textquotesingle}% Upright quotes in Pygmentized code
    }
    \usepackage{upquote} % Upright quotes for verbatim code
    \usepackage{eurosym} % defines \euro
    \usepackage[mathletters]{ucs} % Extended unicode (utf-8) support
    \usepackage{fancyvrb} % verbatim replacement that allows latex
    \usepackage{grffile} % extends the file name processing of package graphics 
                         % to support a larger range
    \makeatletter % fix for old versions of grffile with XeLaTeX
    \@ifpackagelater{grffile}{2019/11/01}
    {
      % Do nothing on new versions
    }
    {
      \def\Gread@@xetex#1{%
        \IfFileExists{"\Gin@base".bb}%
        {\Gread@eps{\Gin@base.bb}}%
        {\Gread@@xetex@aux#1}%
      }
    }
    \makeatother
    \usepackage[Export]{adjustbox} % Used to constrain images to a maximum size
    \adjustboxset{max size={0.9\linewidth}{0.9\paperheight}}

    % The hyperref package gives us a pdf with properly built
    % internal navigation ('pdf bookmarks' for the table of contents,
    % internal cross-reference links, web links for URLs, etc.)
    \usepackage{hyperref}
    % The default LaTeX title has an obnoxious amount of whitespace. By default,
    % titling removes some of it. It also provides customization options.
    \usepackage{titling}
    \usepackage{longtable} % longtable support required by pandoc >1.10
    \usepackage{booktabs}  % table support for pandoc > 1.12.2
    \usepackage[inline]{enumitem} % IRkernel/repr support (it uses the enumerate* environment)
    \usepackage[normalem]{ulem} % ulem is needed to support strikethroughs (\sout)
                                % normalem makes italics be italics, not underlines
    \usepackage{mathrsfs}
    

    
    % Colors for the hyperref package
    \definecolor{urlcolor}{rgb}{0,.145,.698}
    \definecolor{linkcolor}{rgb}{.71,0.21,0.01}
    \definecolor{citecolor}{rgb}{.12,.54,.11}

    % ANSI colors
    \definecolor{ansi-black}{HTML}{3E424D}
    \definecolor{ansi-black-intense}{HTML}{282C36}
    \definecolor{ansi-red}{HTML}{E75C58}
    \definecolor{ansi-red-intense}{HTML}{B22B31}
    \definecolor{ansi-green}{HTML}{00A250}
    \definecolor{ansi-green-intense}{HTML}{007427}
    \definecolor{ansi-yellow}{HTML}{DDB62B}
    \definecolor{ansi-yellow-intense}{HTML}{B27D12}
    \definecolor{ansi-blue}{HTML}{208FFB}
    \definecolor{ansi-blue-intense}{HTML}{0065CA}
    \definecolor{ansi-magenta}{HTML}{D160C4}
    \definecolor{ansi-magenta-intense}{HTML}{A03196}
    \definecolor{ansi-cyan}{HTML}{60C6C8}
    \definecolor{ansi-cyan-intense}{HTML}{258F8F}
    \definecolor{ansi-white}{HTML}{C5C1B4}
    \definecolor{ansi-white-intense}{HTML}{A1A6B2}
    \definecolor{ansi-default-inverse-fg}{HTML}{FFFFFF}
    \definecolor{ansi-default-inverse-bg}{HTML}{000000}

    % common color for the border for error outputs.
    \definecolor{outerrorbackground}{HTML}{FFDFDF}

    % commands and environments needed by pandoc snippets
    % extracted from the output of `pandoc -s`
    \providecommand{\tightlist}{%
      \setlength{\itemsep}{0pt}\setlength{\parskip}{0pt}}
    \DefineVerbatimEnvironment{Highlighting}{Verbatim}{commandchars=\\\{\}}
    % Add ',fontsize=\small' for more characters per line
    \newenvironment{Shaded}{}{}
    \newcommand{\KeywordTok}[1]{\textcolor[rgb]{0.00,0.44,0.13}{\textbf{{#1}}}}
    \newcommand{\DataTypeTok}[1]{\textcolor[rgb]{0.56,0.13,0.00}{{#1}}}
    \newcommand{\DecValTok}[1]{\textcolor[rgb]{0.25,0.63,0.44}{{#1}}}
    \newcommand{\BaseNTok}[1]{\textcolor[rgb]{0.25,0.63,0.44}{{#1}}}
    \newcommand{\FloatTok}[1]{\textcolor[rgb]{0.25,0.63,0.44}{{#1}}}
    \newcommand{\CharTok}[1]{\textcolor[rgb]{0.25,0.44,0.63}{{#1}}}
    \newcommand{\StringTok}[1]{\textcolor[rgb]{0.25,0.44,0.63}{{#1}}}
    \newcommand{\CommentTok}[1]{\textcolor[rgb]{0.38,0.63,0.69}{\textit{{#1}}}}
    \newcommand{\OtherTok}[1]{\textcolor[rgb]{0.00,0.44,0.13}{{#1}}}
    \newcommand{\AlertTok}[1]{\textcolor[rgb]{1.00,0.00,0.00}{\textbf{{#1}}}}
    \newcommand{\FunctionTok}[1]{\textcolor[rgb]{0.02,0.16,0.49}{{#1}}}
    \newcommand{\RegionMarkerTok}[1]{{#1}}
    \newcommand{\ErrorTok}[1]{\textcolor[rgb]{1.00,0.00,0.00}{\textbf{{#1}}}}
    \newcommand{\NormalTok}[1]{{#1}}
    
    % Additional commands for more recent versions of Pandoc
    \newcommand{\ConstantTok}[1]{\textcolor[rgb]{0.53,0.00,0.00}{{#1}}}
    \newcommand{\SpecialCharTok}[1]{\textcolor[rgb]{0.25,0.44,0.63}{{#1}}}
    \newcommand{\VerbatimStringTok}[1]{\textcolor[rgb]{0.25,0.44,0.63}{{#1}}}
    \newcommand{\SpecialStringTok}[1]{\textcolor[rgb]{0.73,0.40,0.53}{{#1}}}
    \newcommand{\ImportTok}[1]{{#1}}
    \newcommand{\DocumentationTok}[1]{\textcolor[rgb]{0.73,0.13,0.13}{\textit{{#1}}}}
    \newcommand{\AnnotationTok}[1]{\textcolor[rgb]{0.38,0.63,0.69}{\textbf{\textit{{#1}}}}}
    \newcommand{\CommentVarTok}[1]{\textcolor[rgb]{0.38,0.63,0.69}{\textbf{\textit{{#1}}}}}
    \newcommand{\VariableTok}[1]{\textcolor[rgb]{0.10,0.09,0.49}{{#1}}}
    \newcommand{\ControlFlowTok}[1]{\textcolor[rgb]{0.00,0.44,0.13}{\textbf{{#1}}}}
    \newcommand{\OperatorTok}[1]{\textcolor[rgb]{0.40,0.40,0.40}{{#1}}}
    \newcommand{\BuiltInTok}[1]{{#1}}
    \newcommand{\ExtensionTok}[1]{{#1}}
    \newcommand{\PreprocessorTok}[1]{\textcolor[rgb]{0.74,0.48,0.00}{{#1}}}
    \newcommand{\AttributeTok}[1]{\textcolor[rgb]{0.49,0.56,0.16}{{#1}}}
    \newcommand{\InformationTok}[1]{\textcolor[rgb]{0.38,0.63,0.69}{\textbf{\textit{{#1}}}}}
    \newcommand{\WarningTok}[1]{\textcolor[rgb]{0.38,0.63,0.69}{\textbf{\textit{{#1}}}}}
    
    
    % Define a nice break command that doesn't care if a line doesn't already
    % exist.
    \def\br{\hspace*{\fill} \\* }
    % Math Jax compatibility definitions
    \def\gt{>}
    \def\lt{<}
    \let\Oldtex\TeX
    \let\Oldlatex\LaTeX
    \renewcommand{\TeX}{\textrm{\Oldtex}}
    \renewcommand{\LaTeX}{\textrm{\Oldlatex}}
    % Document parameters
    % Document title
    \title{Final Course Project}
    
    
    
    
    
% Pygments definitions
\makeatletter
\def\PY@reset{\let\PY@it=\relax \let\PY@bf=\relax%
    \let\PY@ul=\relax \let\PY@tc=\relax%
    \let\PY@bc=\relax \let\PY@ff=\relax}
\def\PY@tok#1{\csname PY@tok@#1\endcsname}
\def\PY@toks#1+{\ifx\relax#1\empty\else%
    \PY@tok{#1}\expandafter\PY@toks\fi}
\def\PY@do#1{\PY@bc{\PY@tc{\PY@ul{%
    \PY@it{\PY@bf{\PY@ff{#1}}}}}}}
\def\PY#1#2{\PY@reset\PY@toks#1+\relax+\PY@do{#2}}

\expandafter\def\csname PY@tok@w\endcsname{\def\PY@tc##1{\textcolor[rgb]{0.73,0.73,0.73}{##1}}}
\expandafter\def\csname PY@tok@c\endcsname{\let\PY@it=\textit\def\PY@tc##1{\textcolor[rgb]{0.25,0.50,0.50}{##1}}}
\expandafter\def\csname PY@tok@cp\endcsname{\def\PY@tc##1{\textcolor[rgb]{0.74,0.48,0.00}{##1}}}
\expandafter\def\csname PY@tok@k\endcsname{\let\PY@bf=\textbf\def\PY@tc##1{\textcolor[rgb]{0.00,0.50,0.00}{##1}}}
\expandafter\def\csname PY@tok@kp\endcsname{\def\PY@tc##1{\textcolor[rgb]{0.00,0.50,0.00}{##1}}}
\expandafter\def\csname PY@tok@kt\endcsname{\def\PY@tc##1{\textcolor[rgb]{0.69,0.00,0.25}{##1}}}
\expandafter\def\csname PY@tok@o\endcsname{\def\PY@tc##1{\textcolor[rgb]{0.40,0.40,0.40}{##1}}}
\expandafter\def\csname PY@tok@ow\endcsname{\let\PY@bf=\textbf\def\PY@tc##1{\textcolor[rgb]{0.67,0.13,1.00}{##1}}}
\expandafter\def\csname PY@tok@nb\endcsname{\def\PY@tc##1{\textcolor[rgb]{0.00,0.50,0.00}{##1}}}
\expandafter\def\csname PY@tok@nf\endcsname{\def\PY@tc##1{\textcolor[rgb]{0.00,0.00,1.00}{##1}}}
\expandafter\def\csname PY@tok@nc\endcsname{\let\PY@bf=\textbf\def\PY@tc##1{\textcolor[rgb]{0.00,0.00,1.00}{##1}}}
\expandafter\def\csname PY@tok@nn\endcsname{\let\PY@bf=\textbf\def\PY@tc##1{\textcolor[rgb]{0.00,0.00,1.00}{##1}}}
\expandafter\def\csname PY@tok@ne\endcsname{\let\PY@bf=\textbf\def\PY@tc##1{\textcolor[rgb]{0.82,0.25,0.23}{##1}}}
\expandafter\def\csname PY@tok@nv\endcsname{\def\PY@tc##1{\textcolor[rgb]{0.10,0.09,0.49}{##1}}}
\expandafter\def\csname PY@tok@no\endcsname{\def\PY@tc##1{\textcolor[rgb]{0.53,0.00,0.00}{##1}}}
\expandafter\def\csname PY@tok@nl\endcsname{\def\PY@tc##1{\textcolor[rgb]{0.63,0.63,0.00}{##1}}}
\expandafter\def\csname PY@tok@ni\endcsname{\let\PY@bf=\textbf\def\PY@tc##1{\textcolor[rgb]{0.60,0.60,0.60}{##1}}}
\expandafter\def\csname PY@tok@na\endcsname{\def\PY@tc##1{\textcolor[rgb]{0.49,0.56,0.16}{##1}}}
\expandafter\def\csname PY@tok@nt\endcsname{\let\PY@bf=\textbf\def\PY@tc##1{\textcolor[rgb]{0.00,0.50,0.00}{##1}}}
\expandafter\def\csname PY@tok@nd\endcsname{\def\PY@tc##1{\textcolor[rgb]{0.67,0.13,1.00}{##1}}}
\expandafter\def\csname PY@tok@s\endcsname{\def\PY@tc##1{\textcolor[rgb]{0.73,0.13,0.13}{##1}}}
\expandafter\def\csname PY@tok@sd\endcsname{\let\PY@it=\textit\def\PY@tc##1{\textcolor[rgb]{0.73,0.13,0.13}{##1}}}
\expandafter\def\csname PY@tok@si\endcsname{\let\PY@bf=\textbf\def\PY@tc##1{\textcolor[rgb]{0.73,0.40,0.53}{##1}}}
\expandafter\def\csname PY@tok@se\endcsname{\let\PY@bf=\textbf\def\PY@tc##1{\textcolor[rgb]{0.73,0.40,0.13}{##1}}}
\expandafter\def\csname PY@tok@sr\endcsname{\def\PY@tc##1{\textcolor[rgb]{0.73,0.40,0.53}{##1}}}
\expandafter\def\csname PY@tok@ss\endcsname{\def\PY@tc##1{\textcolor[rgb]{0.10,0.09,0.49}{##1}}}
\expandafter\def\csname PY@tok@sx\endcsname{\def\PY@tc##1{\textcolor[rgb]{0.00,0.50,0.00}{##1}}}
\expandafter\def\csname PY@tok@m\endcsname{\def\PY@tc##1{\textcolor[rgb]{0.40,0.40,0.40}{##1}}}
\expandafter\def\csname PY@tok@gh\endcsname{\let\PY@bf=\textbf\def\PY@tc##1{\textcolor[rgb]{0.00,0.00,0.50}{##1}}}
\expandafter\def\csname PY@tok@gu\endcsname{\let\PY@bf=\textbf\def\PY@tc##1{\textcolor[rgb]{0.50,0.00,0.50}{##1}}}
\expandafter\def\csname PY@tok@gd\endcsname{\def\PY@tc##1{\textcolor[rgb]{0.63,0.00,0.00}{##1}}}
\expandafter\def\csname PY@tok@gi\endcsname{\def\PY@tc##1{\textcolor[rgb]{0.00,0.63,0.00}{##1}}}
\expandafter\def\csname PY@tok@gr\endcsname{\def\PY@tc##1{\textcolor[rgb]{1.00,0.00,0.00}{##1}}}
\expandafter\def\csname PY@tok@ge\endcsname{\let\PY@it=\textit}
\expandafter\def\csname PY@tok@gs\endcsname{\let\PY@bf=\textbf}
\expandafter\def\csname PY@tok@gp\endcsname{\let\PY@bf=\textbf\def\PY@tc##1{\textcolor[rgb]{0.00,0.00,0.50}{##1}}}
\expandafter\def\csname PY@tok@go\endcsname{\def\PY@tc##1{\textcolor[rgb]{0.53,0.53,0.53}{##1}}}
\expandafter\def\csname PY@tok@gt\endcsname{\def\PY@tc##1{\textcolor[rgb]{0.00,0.27,0.87}{##1}}}
\expandafter\def\csname PY@tok@err\endcsname{\def\PY@bc##1{\setlength{\fboxsep}{0pt}\fcolorbox[rgb]{1.00,0.00,0.00}{1,1,1}{\strut ##1}}}
\expandafter\def\csname PY@tok@kc\endcsname{\let\PY@bf=\textbf\def\PY@tc##1{\textcolor[rgb]{0.00,0.50,0.00}{##1}}}
\expandafter\def\csname PY@tok@kd\endcsname{\let\PY@bf=\textbf\def\PY@tc##1{\textcolor[rgb]{0.00,0.50,0.00}{##1}}}
\expandafter\def\csname PY@tok@kn\endcsname{\let\PY@bf=\textbf\def\PY@tc##1{\textcolor[rgb]{0.00,0.50,0.00}{##1}}}
\expandafter\def\csname PY@tok@kr\endcsname{\let\PY@bf=\textbf\def\PY@tc##1{\textcolor[rgb]{0.00,0.50,0.00}{##1}}}
\expandafter\def\csname PY@tok@bp\endcsname{\def\PY@tc##1{\textcolor[rgb]{0.00,0.50,0.00}{##1}}}
\expandafter\def\csname PY@tok@fm\endcsname{\def\PY@tc##1{\textcolor[rgb]{0.00,0.00,1.00}{##1}}}
\expandafter\def\csname PY@tok@vc\endcsname{\def\PY@tc##1{\textcolor[rgb]{0.10,0.09,0.49}{##1}}}
\expandafter\def\csname PY@tok@vg\endcsname{\def\PY@tc##1{\textcolor[rgb]{0.10,0.09,0.49}{##1}}}
\expandafter\def\csname PY@tok@vi\endcsname{\def\PY@tc##1{\textcolor[rgb]{0.10,0.09,0.49}{##1}}}
\expandafter\def\csname PY@tok@vm\endcsname{\def\PY@tc##1{\textcolor[rgb]{0.10,0.09,0.49}{##1}}}
\expandafter\def\csname PY@tok@sa\endcsname{\def\PY@tc##1{\textcolor[rgb]{0.73,0.13,0.13}{##1}}}
\expandafter\def\csname PY@tok@sb\endcsname{\def\PY@tc##1{\textcolor[rgb]{0.73,0.13,0.13}{##1}}}
\expandafter\def\csname PY@tok@sc\endcsname{\def\PY@tc##1{\textcolor[rgb]{0.73,0.13,0.13}{##1}}}
\expandafter\def\csname PY@tok@dl\endcsname{\def\PY@tc##1{\textcolor[rgb]{0.73,0.13,0.13}{##1}}}
\expandafter\def\csname PY@tok@s2\endcsname{\def\PY@tc##1{\textcolor[rgb]{0.73,0.13,0.13}{##1}}}
\expandafter\def\csname PY@tok@sh\endcsname{\def\PY@tc##1{\textcolor[rgb]{0.73,0.13,0.13}{##1}}}
\expandafter\def\csname PY@tok@s1\endcsname{\def\PY@tc##1{\textcolor[rgb]{0.73,0.13,0.13}{##1}}}
\expandafter\def\csname PY@tok@mb\endcsname{\def\PY@tc##1{\textcolor[rgb]{0.40,0.40,0.40}{##1}}}
\expandafter\def\csname PY@tok@mf\endcsname{\def\PY@tc##1{\textcolor[rgb]{0.40,0.40,0.40}{##1}}}
\expandafter\def\csname PY@tok@mh\endcsname{\def\PY@tc##1{\textcolor[rgb]{0.40,0.40,0.40}{##1}}}
\expandafter\def\csname PY@tok@mi\endcsname{\def\PY@tc##1{\textcolor[rgb]{0.40,0.40,0.40}{##1}}}
\expandafter\def\csname PY@tok@il\endcsname{\def\PY@tc##1{\textcolor[rgb]{0.40,0.40,0.40}{##1}}}
\expandafter\def\csname PY@tok@mo\endcsname{\def\PY@tc##1{\textcolor[rgb]{0.40,0.40,0.40}{##1}}}
\expandafter\def\csname PY@tok@ch\endcsname{\let\PY@it=\textit\def\PY@tc##1{\textcolor[rgb]{0.25,0.50,0.50}{##1}}}
\expandafter\def\csname PY@tok@cm\endcsname{\let\PY@it=\textit\def\PY@tc##1{\textcolor[rgb]{0.25,0.50,0.50}{##1}}}
\expandafter\def\csname PY@tok@cpf\endcsname{\let\PY@it=\textit\def\PY@tc##1{\textcolor[rgb]{0.25,0.50,0.50}{##1}}}
\expandafter\def\csname PY@tok@c1\endcsname{\let\PY@it=\textit\def\PY@tc##1{\textcolor[rgb]{0.25,0.50,0.50}{##1}}}
\expandafter\def\csname PY@tok@cs\endcsname{\let\PY@it=\textit\def\PY@tc##1{\textcolor[rgb]{0.25,0.50,0.50}{##1}}}

\def\PYZbs{\char`\\}
\def\PYZus{\char`\_}
\def\PYZob{\char`\{}
\def\PYZcb{\char`\}}
\def\PYZca{\char`\^}
\def\PYZam{\char`\&}
\def\PYZlt{\char`\<}
\def\PYZgt{\char`\>}
\def\PYZsh{\char`\#}
\def\PYZpc{\char`\%}
\def\PYZdl{\char`\$}
\def\PYZhy{\char`\-}
\def\PYZsq{\char`\'}
\def\PYZdq{\char`\"}
\def\PYZti{\char`\~}
% for compatibility with earlier versions
\def\PYZat{@}
\def\PYZlb{[}
\def\PYZrb{]}
\makeatother


    % For linebreaks inside Verbatim environment from package fancyvrb. 
    \makeatletter
        \newbox\Wrappedcontinuationbox 
        \newbox\Wrappedvisiblespacebox 
        \newcommand*\Wrappedvisiblespace {\textcolor{red}{\textvisiblespace}} 
        \newcommand*\Wrappedcontinuationsymbol {\textcolor{red}{\llap{\tiny$\m@th\hookrightarrow$}}} 
        \newcommand*\Wrappedcontinuationindent {3ex } 
        \newcommand*\Wrappedafterbreak {\kern\Wrappedcontinuationindent\copy\Wrappedcontinuationbox} 
        % Take advantage of the already applied Pygments mark-up to insert 
        % potential linebreaks for TeX processing. 
        %        {, <, #, %, $, ' and ": go to next line. 
        %        _, }, ^, &, >, - and ~: stay at end of broken line. 
        % Use of \textquotesingle for straight quote. 
        \newcommand*\Wrappedbreaksatspecials {% 
            \def\PYGZus{\discretionary{\char`\_}{\Wrappedafterbreak}{\char`\_}}% 
            \def\PYGZob{\discretionary{}{\Wrappedafterbreak\char`\{}{\char`\{}}% 
            \def\PYGZcb{\discretionary{\char`\}}{\Wrappedafterbreak}{\char`\}}}% 
            \def\PYGZca{\discretionary{\char`\^}{\Wrappedafterbreak}{\char`\^}}% 
            \def\PYGZam{\discretionary{\char`\&}{\Wrappedafterbreak}{\char`\&}}% 
            \def\PYGZlt{\discretionary{}{\Wrappedafterbreak\char`\<}{\char`\<}}% 
            \def\PYGZgt{\discretionary{\char`\>}{\Wrappedafterbreak}{\char`\>}}% 
            \def\PYGZsh{\discretionary{}{\Wrappedafterbreak\char`\#}{\char`\#}}% 
            \def\PYGZpc{\discretionary{}{\Wrappedafterbreak\char`\%}{\char`\%}}% 
            \def\PYGZdl{\discretionary{}{\Wrappedafterbreak\char`\$}{\char`\$}}% 
            \def\PYGZhy{\discretionary{\char`\-}{\Wrappedafterbreak}{\char`\-}}% 
            \def\PYGZsq{\discretionary{}{\Wrappedafterbreak\textquotesingle}{\textquotesingle}}% 
            \def\PYGZdq{\discretionary{}{\Wrappedafterbreak\char`\"}{\char`\"}}% 
            \def\PYGZti{\discretionary{\char`\~}{\Wrappedafterbreak}{\char`\~}}% 
        } 
        % Some characters . , ; ? ! / are not pygmentized. 
        % This macro makes them "active" and they will insert potential linebreaks 
        \newcommand*\Wrappedbreaksatpunct {% 
            \lccode`\~`\.\lowercase{\def~}{\discretionary{\hbox{\char`\.}}{\Wrappedafterbreak}{\hbox{\char`\.}}}% 
            \lccode`\~`\,\lowercase{\def~}{\discretionary{\hbox{\char`\,}}{\Wrappedafterbreak}{\hbox{\char`\,}}}% 
            \lccode`\~`\;\lowercase{\def~}{\discretionary{\hbox{\char`\;}}{\Wrappedafterbreak}{\hbox{\char`\;}}}% 
            \lccode`\~`\:\lowercase{\def~}{\discretionary{\hbox{\char`\:}}{\Wrappedafterbreak}{\hbox{\char`\:}}}% 
            \lccode`\~`\?\lowercase{\def~}{\discretionary{\hbox{\char`\?}}{\Wrappedafterbreak}{\hbox{\char`\?}}}% 
            \lccode`\~`\!\lowercase{\def~}{\discretionary{\hbox{\char`\!}}{\Wrappedafterbreak}{\hbox{\char`\!}}}% 
            \lccode`\~`\/\lowercase{\def~}{\discretionary{\hbox{\char`\/}}{\Wrappedafterbreak}{\hbox{\char`\/}}}% 
            \catcode`\.\active
            \catcode`\,\active 
            \catcode`\;\active
            \catcode`\:\active
            \catcode`\?\active
            \catcode`\!\active
            \catcode`\/\active 
            \lccode`\~`\~ 	
        }
    \makeatother

    \let\OriginalVerbatim=\Verbatim
    \makeatletter
    \renewcommand{\Verbatim}[1][1]{%
        %\parskip\z@skip
        \sbox\Wrappedcontinuationbox {\Wrappedcontinuationsymbol}%
        \sbox\Wrappedvisiblespacebox {\FV@SetupFont\Wrappedvisiblespace}%
        \def\FancyVerbFormatLine ##1{\hsize\linewidth
            \vtop{\raggedright\hyphenpenalty\z@\exhyphenpenalty\z@
                \doublehyphendemerits\z@\finalhyphendemerits\z@
                \strut ##1\strut}%
        }%
        % If the linebreak is at a space, the latter will be displayed as visible
        % space at end of first line, and a continuation symbol starts next line.
        % Stretch/shrink are however usually zero for typewriter font.
        \def\FV@Space {%
            \nobreak\hskip\z@ plus\fontdimen3\font minus\fontdimen4\font
            \discretionary{\copy\Wrappedvisiblespacebox}{\Wrappedafterbreak}
            {\kern\fontdimen2\font}%
        }%
        
        % Allow breaks at special characters using \PYG... macros.
        \Wrappedbreaksatspecials
        % Breaks at punctuation characters . , ; ? ! and / need catcode=\active 	
        \OriginalVerbatim[#1,codes*=\Wrappedbreaksatpunct]%
    }
    \makeatother

    % Exact colors from NB
    \definecolor{incolor}{HTML}{303F9F}
    \definecolor{outcolor}{HTML}{D84315}
    \definecolor{cellborder}{HTML}{CFCFCF}
    \definecolor{cellbackground}{HTML}{F7F7F7}
    
    % prompt
    \makeatletter
    \newcommand{\boxspacing}{\kern\kvtcb@left@rule\kern\kvtcb@boxsep}
    \makeatother
    \newcommand{\prompt}[4]{
        {\ttfamily\llap{{\color{#2}[#3]:\hspace{3pt}#4}}\vspace{-\baselineskip}}
    }
    

    
    % Prevent overflowing lines due to hard-to-break entities
    \sloppy 
    % Setup hyperref package
    \hypersetup{
      breaklinks=true,  % so long urls are correctly broken across lines
      colorlinks=true,
      urlcolor=urlcolor,
      linkcolor=linkcolor,
      citecolor=citecolor,
      }
    % Slightly bigger margins than the latex defaults
    
    \geometry{verbose,tmargin=1in,bmargin=1in,lmargin=1in,rmargin=1in}
    
    

\begin{document}
    
    \maketitle
    
    

    
    \begin{tcolorbox}[breakable, size=fbox, boxrule=1pt, pad at break*=1mm,colback=cellbackground, colframe=cellborder]
\prompt{In}{incolor}{1}{\boxspacing}
\begin{Verbatim}[commandchars=\\\{\}]
\PY{k+kn}{import} \PY{n+nn}{pandas} \PY{k}{as} \PY{n+nn}{pd}
\PY{k+kn}{import} \PY{n+nn}{numpy} \PY{k}{as} \PY{n+nn}{np}
\PY{k+kn}{from} \PY{n+nn}{sklearn}\PY{n+nn}{.}\PY{n+nn}{preprocessing} \PY{k+kn}{import} \PY{n}{StandardScaler}
\PY{k+kn}{import} \PY{n+nn}{plotly}\PY{n+nn}{.}\PY{n+nn}{express} \PY{k}{as} \PY{n+nn}{px}
\end{Verbatim}
\end{tcolorbox}

    Во прилог ќе бидат практично применети концептите за намалување на
димензионалноста со помош на Principal Component Analysis (PCA).

    \hypertarget{ux434ux435ux444ux438ux43dux438ux440ux430ux45aux435-ux43dux430-ux43fux43eux434ux430ux442ux43eux447ux43dux43eux442ux43e-ux43cux43dux43eux436ux435ux441ux442ux432ux43e}{%
\section{Дефинирање на податочното
множество}\label{ux434ux435ux444ux438ux43dux438ux440ux430ux45aux435-ux43dux430-ux43fux43eux434ux430ux442ux43eux447ux43dux43eux442ux43e-ux43cux43dux43eux436ux435ux441ux442ux432ux43e}}

    Податочното множество кое ќе биде користено е Heart Disease UCI,
достапно на следниот
\href{https://www.kaggle.com/ronitf/heart-disease-uci}{линк}. Станува
збор за податочно множество со \texttt{14} променливи(карактеристики) па
векторите на нашите податоци се од \texttt{14} димензионален векторски
простор.

Наша цел е да ги визуелизираме овие податоци, со цел увидеме дали
постојат повеќе кластери на испитаници. Но, за да направиме
визуелизација мора да имаме најмногу \texttt{3} димензии.

    \begin{tcolorbox}[breakable, size=fbox, boxrule=1pt, pad at break*=1mm,colback=cellbackground, colframe=cellborder]
\prompt{In}{incolor}{2}{\boxspacing}
\begin{Verbatim}[commandchars=\\\{\}]
\PY{n}{dataset} \PY{o}{=} \PY{n}{pd}\PY{o}{.}\PY{n}{read\PYZus{}csv}\PY{p}{(}\PY{l+s+s1}{\PYZsq{}}\PY{l+s+s1}{datasets/heart.csv}\PY{l+s+s1}{\PYZsq{}}\PY{p}{)}
\PY{n}{dataset}\PY{o}{.}\PY{n}{head}\PY{p}{(}\PY{p}{)}
\end{Verbatim}
\end{tcolorbox}

            \begin{tcolorbox}[breakable, size=fbox, boxrule=.5pt, pad at break*=1mm, opacityfill=0]
\prompt{Out}{outcolor}{2}{\boxspacing}
\begin{Verbatim}[commandchars=\\\{\}]
   age  sex  cp  trestbps  chol  fbs  restecg  thalach  exang  oldpeak  slope  \textbackslash{}
0   63    1   3       145   233    1        0      150      0      2.3      0
1   37    1   2       130   250    0        1      187      0      3.5      0
2   41    0   1       130   204    0        0      172      0      1.4      2
3   56    1   1       120   236    0        1      178      0      0.8      2
4   57    0   0       120   354    0        1      163      1      0.6      2

   ca  thal  target
0   0     1       1
1   0     2       1
2   0     2       1
3   0     2       1
4   0     2       1
\end{Verbatim}
\end{tcolorbox}
        
    Во нашето податочно множество, немаме променливи(карактеристики) за кои
недостасуваат вредности за одредени примероци.

    \begin{tcolorbox}[breakable, size=fbox, boxrule=1pt, pad at break*=1mm,colback=cellbackground, colframe=cellborder]
\prompt{In}{incolor}{3}{\boxspacing}
\begin{Verbatim}[commandchars=\\\{\}]
\PY{n}{dataset}\PY{o}{.}\PY{n}{isna}\PY{p}{(}\PY{p}{)}\PY{o}{.}\PY{n}{any}\PY{p}{(}\PY{p}{)}\PY{o}{.}\PY{n}{any}\PY{p}{(}\PY{p}{)}
\end{Verbatim}
\end{tcolorbox}

            \begin{tcolorbox}[breakable, size=fbox, boxrule=.5pt, pad at break*=1mm, opacityfill=0]
\prompt{Out}{outcolor}{3}{\boxspacing}
\begin{Verbatim}[commandchars=\\\{\}]
False
\end{Verbatim}
\end{tcolorbox}
        
    Сите вредности се нумерички, па нема потреба од нивно енкодирање.

    \begin{tcolorbox}[breakable, size=fbox, boxrule=1pt, pad at break*=1mm,colback=cellbackground, colframe=cellborder]
\prompt{In}{incolor}{4}{\boxspacing}
\begin{Verbatim}[commandchars=\\\{\}]
\PY{n}{dataset}\PY{o}{.}\PY{n}{info}\PY{p}{(}\PY{p}{)}
\end{Verbatim}
\end{tcolorbox}

    \begin{Verbatim}[commandchars=\\\{\}]
<class 'pandas.core.frame.DataFrame'>
RangeIndex: 303 entries, 0 to 302
Data columns (total 14 columns):
 \#   Column    Non-Null Count  Dtype
---  ------    --------------  -----
 0   age       303 non-null    int64
 1   sex       303 non-null    int64
 2   cp        303 non-null    int64
 3   trestbps  303 non-null    int64
 4   chol      303 non-null    int64
 5   fbs       303 non-null    int64
 6   restecg   303 non-null    int64
 7   thalach   303 non-null    int64
 8   exang     303 non-null    int64
 9   oldpeak   303 non-null    float64
 10  slope     303 non-null    int64
 11  ca        303 non-null    int64
 12  thal      303 non-null    int64
 13  target    303 non-null    int64
dtypes: float64(1), int64(13)
memory usage: 33.3 KB
    \end{Verbatim}

    Податоците претставени како матрица:

    \begin{tcolorbox}[breakable, size=fbox, boxrule=1pt, pad at break*=1mm,colback=cellbackground, colframe=cellborder]
\prompt{In}{incolor}{5}{\boxspacing}
\begin{Verbatim}[commandchars=\\\{\}]
\PY{n}{dataset}\PY{o}{.}\PY{n}{values}
\end{Verbatim}
\end{tcolorbox}

            \begin{tcolorbox}[breakable, size=fbox, boxrule=.5pt, pad at break*=1mm, opacityfill=0]
\prompt{Out}{outcolor}{5}{\boxspacing}
\begin{Verbatim}[commandchars=\\\{\}]
array([[63.,  1.,  3., {\ldots},  0.,  1.,  1.],
       [37.,  1.,  2., {\ldots},  0.,  2.,  1.],
       [41.,  0.,  1., {\ldots},  0.,  2.,  1.],
       {\ldots},
       [68.,  1.,  0., {\ldots},  2.,  3.,  0.],
       [57.,  1.,  0., {\ldots},  1.,  3.,  0.],
       [57.,  0.,  1., {\ldots},  1.,  2.,  0.]])
\end{Verbatim}
\end{tcolorbox}
        
    \hypertarget{ux447ux435ux43aux43eux440ux438-ux43eux434-principal-component-analysis}{%
\section{Чекори од Principal Component
Analysis}\label{ux447ux435ux43aux43eux440ux438-ux43eux434-principal-component-analysis}}

    \hypertarget{ux447ux435ux43aux43eux440-1-ux441ux442ux430ux43dux434ux430ux440ux434ux438ux437ux430ux446ux438ux458ux430}{%
\subsection{Чекор 1:
Стандардизација}\label{ux447ux435ux43aux43eux440-1-ux441ux442ux430ux43dux434ux430ux440ux434ux438ux437ux430ux446ux438ux458ux430}}

    \begin{tcolorbox}[breakable, size=fbox, boxrule=1pt, pad at break*=1mm,colback=cellbackground, colframe=cellborder]
\prompt{In}{incolor}{6}{\boxspacing}
\begin{Verbatim}[commandchars=\\\{\}]
\PY{n}{scaler} \PY{o}{=} \PY{n}{StandardScaler}\PY{p}{(}\PY{p}{)}

\PY{n}{standardized\PYZus{}dataset} \PY{o}{=} \PY{n}{pd}\PY{o}{.}\PY{n}{DataFrame}\PY{p}{(}\PY{n}{scaler}\PY{o}{.}\PY{n}{fit\PYZus{}transform}\PY{p}{(}\PY{n}{dataset}\PY{o}{.}\PY{n}{values}\PY{p}{)}\PY{p}{)}
\PY{n}{standardized\PYZus{}dataset}\PY{o}{.}\PY{n}{columns} \PY{o}{=} \PY{n}{dataset}\PY{o}{.}\PY{n}{columns}
\end{Verbatim}
\end{tcolorbox}

    \begin{tcolorbox}[breakable, size=fbox, boxrule=1pt, pad at break*=1mm,colback=cellbackground, colframe=cellborder]
\prompt{In}{incolor}{7}{\boxspacing}
\begin{Verbatim}[commandchars=\\\{\}]
\PY{n}{standardized\PYZus{}dataset}\PY{o}{.}\PY{n}{head}\PY{p}{(}\PY{p}{)}
\end{Verbatim}
\end{tcolorbox}

            \begin{tcolorbox}[breakable, size=fbox, boxrule=.5pt, pad at break*=1mm, opacityfill=0]
\prompt{Out}{outcolor}{7}{\boxspacing}
\begin{Verbatim}[commandchars=\\\{\}]
        age       sex        cp  trestbps      chol       fbs   restecg  \textbackslash{}
0  0.952197  0.681005  1.973123  0.763956 -0.256334  2.394438 -1.005832
1 -1.915313  0.681005  1.002577 -0.092738  0.072199 -0.417635  0.898962
2 -1.474158 -1.468418  0.032031 -0.092738 -0.816773 -0.417635 -1.005832
3  0.180175  0.681005  0.032031 -0.663867 -0.198357 -0.417635  0.898962
4  0.290464 -1.468418 -0.938515 -0.663867  2.082050 -0.417635  0.898962

    thalach     exang   oldpeak     slope        ca      thal    target
0  0.015443 -0.696631  1.087338 -2.274579 -0.714429 -2.148873  0.914529
1  1.633471 -0.696631  2.122573 -2.274579 -0.714429 -0.512922  0.914529
2  0.977514 -0.696631  0.310912  0.976352 -0.714429 -0.512922  0.914529
3  1.239897 -0.696631 -0.206705  0.976352 -0.714429 -0.512922  0.914529
4  0.583939  1.435481 -0.379244  0.976352 -0.714429 -0.512922  0.914529
\end{Verbatim}
\end{tcolorbox}
        
    \hypertarget{ux447ux435ux43aux43eux440-2-ux43fux440ux435ux441ux43cux435ux442ux43aux430-ux43dux430-ux43cux430ux442ux440ux438ux446ux430-ux43dux430-ux43aux43eux432ux430ux440ux438ux458ux430ux43dux441ux438}{%
\subsection{Чекор 2: Пресметка на матрица на
коваријанси}\label{ux447ux435ux43aux43eux440-2-ux43fux440ux435ux441ux43cux435ux442ux43aux430-ux43dux430-ux43cux430ux442ux440ux438ux446ux430-ux43dux430-ux43aux43eux432ux430ux440ux438ux458ux430ux43dux441ux438}}

    \begin{tcolorbox}[breakable, size=fbox, boxrule=1pt, pad at break*=1mm,colback=cellbackground, colframe=cellborder]
\prompt{In}{incolor}{8}{\boxspacing}
\begin{Verbatim}[commandchars=\\\{\}]
\PY{n}{standardized\PYZus{}dataset}\PY{o}{.}\PY{n}{cov}\PY{p}{(}\PY{p}{)}
\end{Verbatim}
\end{tcolorbox}

            \begin{tcolorbox}[breakable, size=fbox, boxrule=.5pt, pad at break*=1mm, opacityfill=0]
\prompt{Out}{outcolor}{8}{\boxspacing}
\begin{Verbatim}[commandchars=\\\{\}]
               age       sex        cp  trestbps      chol       fbs  \textbackslash{}
age       1.003311 -0.098773 -0.068880  0.280276  0.214385  0.121709
sex      -0.098773  1.003311 -0.049516 -0.056957 -0.198568  0.045181
cp       -0.068880 -0.049516  1.003311  0.047765 -0.077159  0.094757
trestbps  0.280276 -0.056957  0.047765  1.003311  0.123582  0.178118
chol      0.214385 -0.198568 -0.077159  0.123582  1.003311  0.013338
fbs       0.121709  0.045181  0.094757  0.178118  0.013338  1.003311
restecg  -0.116596 -0.058389  0.044568 -0.114481 -0.151540 -0.084468
thalach  -0.399842 -0.044166  0.296741 -0.046852 -0.009973 -0.008595
exang     0.097121  0.142133 -0.395586  0.067840  0.067245  0.025750
oldpeak   0.210708  0.096411 -0.149724  0.193856  0.054131  0.005766
slope    -0.169373 -0.030812  0.120113 -0.121877 -0.004051 -0.060093
ca        0.277241  0.118653 -0.181653  0.101725  0.070744  0.138436
thal      0.068227  0.210737 -0.162271  0.062416  0.099130 -0.032125
target   -0.226185 -0.281867  0.435235 -0.145411 -0.085521 -0.028139

           restecg   thalach     exang   oldpeak     slope        ca  \textbackslash{}
age      -0.116596 -0.399842  0.097121  0.210708 -0.169373  0.277241
sex      -0.058389 -0.044166  0.142133  0.096411 -0.030812  0.118653
cp        0.044568  0.296741 -0.395586 -0.149724  0.120113 -0.181653
trestbps -0.114481 -0.046852  0.067840  0.193856 -0.121877  0.101725
chol     -0.151540 -0.009973  0.067245  0.054131 -0.004051  0.070744
fbs      -0.084468 -0.008595  0.025750  0.005766 -0.060093  0.138436
restecg   1.003311  0.044270 -0.070967 -0.058965  0.093353 -0.072281
thalach   0.044270  1.003311 -0.380066 -0.345327  0.388065 -0.213883
exang    -0.070967 -0.380066  1.003311  0.289177 -0.258602  0.116123
oldpeak  -0.058965 -0.345327  0.289177  1.003311 -0.579449  0.223420
slope     0.093353  0.388065 -0.258602 -0.579449  1.003311 -0.080421
ca       -0.072281 -0.213883  0.116123  0.223420 -0.080421  1.003311
thal     -0.012021 -0.096758  0.207438  0.210940 -0.105111  0.152335
target    0.137684  0.423137 -0.438203 -0.432122  0.347022 -0.393021

              thal    target
age       0.068227 -0.226185
sex       0.210737 -0.281867
cp       -0.162271  0.435235
trestbps  0.062416 -0.145411
chol      0.099130 -0.085521
fbs      -0.032125 -0.028139
restecg  -0.012021  0.137684
thalach  -0.096758  0.423137
exang     0.207438 -0.438203
oldpeak   0.210940 -0.432122
slope    -0.105111  0.347022
ca        0.152335 -0.393021
thal      1.003311 -0.345168
target   -0.345168  1.003311
\end{Verbatim}
\end{tcolorbox}
        
    \begin{tcolorbox}[breakable, size=fbox, boxrule=1pt, pad at break*=1mm,colback=cellbackground, colframe=cellborder]
\prompt{In}{incolor}{9}{\boxspacing}
\begin{Verbatim}[commandchars=\\\{\}]
\PY{n}{cov\PYZus{}matrix} \PY{o}{=} \PY{n}{np}\PY{o}{.}\PY{n}{matrix}\PY{p}{(}\PY{n}{standardized\PYZus{}dataset}\PY{o}{.}\PY{n}{cov}\PY{p}{(}\PY{p}{)}\PY{o}{.}\PY{n}{to\PYZus{}numpy}\PY{p}{(}\PY{p}{)}\PY{p}{)}
\end{Verbatim}
\end{tcolorbox}

    \hypertarget{ux447ux435ux43aux43eux440-3-ux43fux440ux435ux441ux43cux435ux442ux43aux430-ux43dux430-ux441ux43eux43fux441ux442ux432ux435ux43dux438-ux432ux440ux435ux434ux43dux43eux441ux442ux438-ux438-ux432ux435ux43aux442ux43eux440ux438-ux43cux430ux442ux440ux438ux446ux430ux442ux430-ux43dux430-ux43aux43eux432ux430ux440ux438jux430ux43dux441ux438}{%
\subsection{Чекор 3: Пресметка на сопствени вредности и вектори
матрицата на
ковариjанси}\label{ux447ux435ux43aux43eux440-3-ux43fux440ux435ux441ux43cux435ux442ux43aux430-ux43dux430-ux441ux43eux43fux441ux442ux432ux435ux43dux438-ux432ux440ux435ux434ux43dux43eux441ux442ux438-ux438-ux432ux435ux43aux442ux43eux440ux438-ux43cux430ux442ux440ux438ux446ux430ux442ux430-ux43dux430-ux43aux43eux432ux430ux440ux438jux430ux43dux441ux438}}

    За пресметка на сопствените вредности ќе користиме \texttt{np.linalg}, и
основна функција за нивно сортирање.

    \begin{tcolorbox}[breakable, size=fbox, boxrule=1pt, pad at break*=1mm,colback=cellbackground, colframe=cellborder]
\prompt{In}{incolor}{10}{\boxspacing}
\begin{Verbatim}[commandchars=\\\{\}]
\PY{n}{eigen\PYZus{}values} \PY{o}{=} \PY{o}{\PYZhy{}}\PY{n}{np}\PY{o}{.}\PY{n}{sort}\PY{p}{(}\PY{o}{\PYZhy{}}\PY{n}{np}\PY{o}{.}\PY{n}{linalg}\PY{o}{.}\PY{n}{eigvals}\PY{p}{(}\PY{n}{cov\PYZus{}matrix}\PY{p}{)}\PY{p}{)}
\end{Verbatim}
\end{tcolorbox}

    \begin{tcolorbox}[breakable, size=fbox, boxrule=1pt, pad at break*=1mm,colback=cellbackground, colframe=cellborder]
\prompt{In}{incolor}{11}{\boxspacing}
\begin{Verbatim}[commandchars=\\\{\}]
\PY{n}{eigen\PYZus{}values}\PY{p}{,} \PY{n}{eigen\PYZus{}vectors} \PY{o}{=} \PY{n}{np}\PY{o}{.}\PY{n}{linalg}\PY{o}{.}\PY{n}{eig}\PY{p}{(}\PY{n}{cov\PYZus{}matrix}\PY{p}{)}

\PY{n}{idx} \PY{o}{=} \PY{n}{eigen\PYZus{}values}\PY{o}{.}\PY{n}{argsort}\PY{p}{(}\PY{p}{)}\PY{p}{[}\PY{p}{:}\PY{p}{:}\PY{o}{\PYZhy{}}\PY{l+m+mi}{1}\PY{p}{]}   
\PY{n}{eigen\PYZus{}values} \PY{o}{=} \PY{n}{eigen\PYZus{}values}\PY{p}{[}\PY{n}{idx}\PY{p}{]}
\PY{n}{eigen\PYZus{}vectors} \PY{o}{=} \PY{n}{eigen\PYZus{}vectors}\PY{p}{[}\PY{p}{:}\PY{p}{,}\PY{n}{idx}\PY{p}{]}
\end{Verbatim}
\end{tcolorbox}

    \hypertarget{ux447ux435ux43aux43eux440-4-ux438ux437ux431ux43eux440-ux43dux430-ux43aux43eux43cux43fux43eux43dux435ux43dux442ux438-ux438-ux434ux43eux431ux438ux432ux430ux45aux435-ux43dux430-feature-ux432ux435ux43aux442ux43eux440}{%
\subsection{Чекор 4: Избор на компоненти и добивање на feature
вектор}\label{ux447ux435ux43aux43eux440-4-ux438ux437ux431ux43eux440-ux43dux430-ux43aux43eux43cux43fux43eux43dux435ux43dux442ux438-ux438-ux434ux43eux431ux438ux432ux430ux45aux435-ux43dux430-feature-ux432ux435ux43aux442ux43eux440}}

    Bo Чекор 4, и Чекор 5, меѓу-резултантните матрици се веќе транспонирани
поради defaults, па за добивање на резултатот ќе ја кориситиме следната
линеарна трансформација

\(FinalDataSet = (FeatureVector \times StandardizedDataSet^T)^T\)

    Како што беше дискутирано претходно, за одбирање на компоненти имаме две
опции, и тоа:

\begin{itemize}
\item
  дефинирање праг на значајност
\item
  априори дефинирање на димензија до која сакаме да ги намалиме
  податоците
\end{itemize}

Во овој пример априори ќе ја дефинираме димензијата, и ќе одбереме таа
да биде \texttt{3}, бидејќи крајната цел е да ги визуелизираме
податоците.

    \begin{tcolorbox}[breakable, size=fbox, boxrule=1pt, pad at break*=1mm,colback=cellbackground, colframe=cellborder]
\prompt{In}{incolor}{12}{\boxspacing}
\begin{Verbatim}[commandchars=\\\{\}]
\PY{n}{TARGET\PYZus{}DIMENSION} \PY{o}{=} \PY{l+m+mi}{3}
\PY{n}{feature\PYZus{}vector} \PY{o}{=} \PY{n}{eigen\PYZus{}vectors}\PY{p}{[}\PY{p}{:}\PY{n}{TARGET\PYZus{}DIMENSION}\PY{p}{]}
\end{Verbatim}
\end{tcolorbox}

    \begin{tcolorbox}[breakable, size=fbox, boxrule=1pt, pad at break*=1mm,colback=cellbackground, colframe=cellborder]
\prompt{In}{incolor}{13}{\boxspacing}
\begin{Verbatim}[commandchars=\\\{\}]
\PY{n}{feature\PYZus{}vector}\PY{o}{.}\PY{n}{shape}
\end{Verbatim}
\end{tcolorbox}

            \begin{tcolorbox}[breakable, size=fbox, boxrule=.5pt, pad at break*=1mm, opacityfill=0]
\prompt{Out}{outcolor}{13}{\boxspacing}
\begin{Verbatim}[commandchars=\\\{\}]
(3, 14)
\end{Verbatim}
\end{tcolorbox}
        
    \hypertarget{ux447ux435ux43aux43eux440-5-ux43dux430ux43cux430ux43bux443ux432ux430ux45aux435-ux43dux430-ux434ux438ux43cux435ux43dux437ux438jux430ux442ux430}{%
\subsection{Чекор 5: Намалување на
димензиjата}\label{ux447ux435ux43aux43eux440-5-ux43dux430ux43cux430ux43bux443ux432ux430ux45aux435-ux43dux430-ux434ux438ux43cux435ux43dux437ux438jux430ux442ux430}}

    \begin{tcolorbox}[breakable, size=fbox, boxrule=1pt, pad at break*=1mm,colback=cellbackground, colframe=cellborder]
\prompt{In}{incolor}{14}{\boxspacing}
\begin{Verbatim}[commandchars=\\\{\}]
\PY{n}{final\PYZus{}dataset} \PY{o}{=} \PY{n}{np}\PY{o}{.}\PY{n}{transpose}\PY{p}{(}\PY{n}{np}\PY{o}{.}\PY{n}{matmul}\PY{p}{(}\PY{n}{feature\PYZus{}vector}\PY{p}{,} \PY{n}{np}\PY{o}{.}\PY{n}{transpose}\PY{p}{(}\PY{n}{standardized\PYZus{}dataset}\PY{p}{)}\PY{p}{)}\PY{p}{)}
\end{Verbatim}
\end{tcolorbox}

    \begin{tcolorbox}[breakable, size=fbox, boxrule=1pt, pad at break*=1mm,colback=cellbackground, colframe=cellborder]
\prompt{In}{incolor}{15}{\boxspacing}
\begin{Verbatim}[commandchars=\\\{\}]
\PY{n}{final\PYZus{}dataset}
\end{Verbatim}
\end{tcolorbox}

            \begin{tcolorbox}[breakable, size=fbox, boxrule=.5pt, pad at break*=1mm, opacityfill=0]
\prompt{Out}{outcolor}{15}{\boxspacing}
\begin{Verbatim}[commandchars=\\\{\}]
            0         1         2
0   -0.286687  0.110460 -1.914001
1   -0.467832  0.684439 -0.893856
2   -1.265855  0.513491  0.352265
3   -0.385241 -0.656087 -0.169573
4   -1.396762  1.160139  0.477217
..        {\ldots}       {\ldots}       {\ldots}
298 -0.416647  0.652530  0.303550
299 -0.376835 -2.061995 -0.526859
300  1.121740  1.366236 -0.713181
301  1.648491  0.125585  0.280630
302  0.047174  0.139822 -0.800181

[303 rows x 3 columns]
\end{Verbatim}
\end{tcolorbox}
        
    \hypertarget{ux432ux438ux437ux443ux435ux43bux438ux437ux430ux446ux438ux458ux430-ux43dux430-ux43fux43eux434ux430ux442ux43eux446ux438ux442ux435}{%
\section{Визуелизација на
податоците}\label{ux432ux438ux437ux443ux435ux43bux438ux437ux430ux446ux438ux458ux430-ux43dux430-ux43fux43eux434ux430ux442ux43eux446ux438ux442ux435}}

    Сега нашите податоци се со помала димензија, односно векторите кои ги
репрезентираат се од \texttt{3} димензионален векторски простор, па
истите можеме да ги визуелизираме.

    \begin{tcolorbox}[breakable, size=fbox, boxrule=1pt, pad at break*=1mm,colback=cellbackground, colframe=cellborder]
\prompt{In}{incolor}{16}{\boxspacing}
\begin{Verbatim}[commandchars=\\\{\}]
\PY{n}{fig} \PY{o}{=} \PY{n}{px}\PY{o}{.}\PY{n}{scatter\PYZus{}3d}\PY{p}{(}\PY{n}{final\PYZus{}dataset}\PY{p}{,} \PY{n}{x}\PY{o}{=}\PY{l+m+mi}{0}\PY{p}{,} \PY{n}{y}\PY{o}{=}\PY{l+m+mi}{1}\PY{p}{,} \PY{n}{z}\PY{o}{=}\PY{l+m+mi}{2}\PY{p}{,} 
                    \PY{n}{title}\PY{o}{=}\PY{l+s+s2}{\PYZdq{}}\PY{l+s+s2}{Визуелизација на податоците во 3Д простор}\PY{l+s+s2}{\PYZdq{}}\PY{p}{,}
                    \PY{n}{labels}\PY{o}{=}\PY{p}{\PYZob{}}\PY{n+nb}{str}\PY{p}{(}\PY{n}{i}\PY{p}{)}\PY{p}{:}\PY{l+s+sa}{f}\PY{l+s+s2}{\PYZdq{}}\PY{l+s+s2}{PC: }\PY{l+s+si}{\PYZob{}}\PY{n}{i}\PY{l+s+si}{\PYZcb{}}\PY{l+s+s2}{\PYZdq{}} \PY{k}{for} \PY{n}{i} \PY{o+ow}{in} \PY{n+nb}{range}\PY{p}{(}\PY{l+m+mi}{0}\PY{p}{,}\PY{n}{TARGET\PYZus{}DIMENSION}\PY{p}{)}\PY{p}{\PYZcb{}}\PY{p}{,}
                    \PY{n}{width}\PY{o}{=}\PY{l+m+mi}{800}\PY{p}{,} 
                    \PY{n}{height}\PY{o}{=}\PY{l+m+mi}{800}
                   \PY{p}{)}
\PY{n}{fig}\PY{o}{.}\PY{n}{show}\PY{p}{(}\PY{p}{)}
\end{Verbatim}
\end{tcolorbox}

    
    
    
    
    \begin{tcolorbox}[breakable, size=fbox, boxrule=1pt, pad at break*=1mm,colback=cellbackground, colframe=cellborder]
\prompt{In}{incolor}{16}{\boxspacing}
\begin{Verbatim}[commandchars=\\\{\}]

\end{Verbatim}
\end{tcolorbox}


    % Add a bibliography block to the postdoc
    
    
    
\end{document}
